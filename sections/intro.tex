%! TEX root = ../main.tex
\documentclass[main]{subfiles}

\begin{document}
\chapter{緒言}
\section{背景}
近年,学術研究や企業での技術文書作成において,効率的かつ統一されたフォーマットでの論文執筆が求められている.
特に理工系分野の論文では,数式や図表を多用するため、{\LaTeX} が主流の文書作成ツールとなって久しい.
しかし,{\LaTeX} の導入には環境構築が必要であり,特に初学者にとっては初期設定が難解で手間がかかることが課題となる.
加えて,分野や投稿誌ごとに異なる書式設定や参考文献のスタイルの調整が必要なため,標準化された設定とテンプレートが求められている.

東京農工大学工学部知能情報システム工学科金子研究室でも,
学生が統一されたフォーマットで効率よく卒業論文や修士論文を執筆するために,このような課題に直面している.
特に,Linux仮想環境を用いてセットアップを自動化することで,技術的な障壁を軽減し,
迅速に論文執筆を開始できる環境の整備が期待されている.

\section{目的}
%%4~5行
本研究では,金子研究室の学生向けに,{\LaTeX} による卒業論文および修士論文の執筆環境を自動構築するシステムを開発することを目的とする.
このシステムは,DockerとVisual Studio CodeのDevContainerを組み合わせた仮想環境を用いることで,
学生が複雑なセットアップ手順なしに即座に執筆を開始できることを目指す.
具体的には,{\TeX}Liveを導入したDockerイメージと,研究室向けにカスタマイズされた{\LaTeX}クラスファイルやテンプレートを提供することで,環境構築を簡略化し,論文執筆の効率化を図る.
また,Bib{\TeX}を用いた参考文献の引用スタイルの自動化や,日本語対応を含むIEEEスタイルのカスタマイズも行い,研究室標準に合致した執筆環境を整備する.

\section{本論文の構成}
以下,本論文は次のとおり構成される.
まず,第2章では,関連研究や既存システムについて述べる.
次に,第3章では,本研究で提案するシステムについて述べる.
第4章では,システムに対する評価実験について述べる.
第5章では,第4章で述べた実験の結果および考察について述べる.
最後に,第6章で,本研究のまとめと今後の展望について述べる.
\end{document}