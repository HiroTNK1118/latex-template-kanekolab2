%! TEX root = ../main.tex
\documentclass[main]{subfiles}

\begin{document}
\chapter{関連研究}
\section{Linux仮想環境での{\TeX}執筆}
Windows環境での{\TeX}文書作成には,Linux仮想環境を利用することで多くの利点が得られる.特に文字コードの統一が図られる点が重要であり,Unicode(UTF-8)形式で統一することで,ファイル間の互換性が向上する.また,Linux上で提供される{\TeX} Liveは,パッケージ管理や更新が容易で,利用するツール群も充実しているため,効率的な{\TeX}執筆環境を構築できる.以下,WindowsマシンにおけるLinux仮想環境の構築方法として,WSL + Ubuntu,Dockerコンテナ,そしてVSCode DevContainerを用いたアプローチについて紹介する.

\section{Linux仮想環境の構築}
Windows上にLinux環境を構築する方法として,WSL + Ubuntu,Docker,およびVSCode DevContainerがある.それぞれの特徴を以下に示す.

\subsection{WSL + Ubuntuによる仮想マシン}
WSL(Windows Subsystem for Linux)は,Microsoftが提供するWindows上でLinuxを仮想的に実行する仕組みである.WSL上でUbuntuなどのLinuxディストリビューションをインストールすることで,Linuxベースの{\TeX} Liveを利用した{\TeX}執筆環境を構築できる.WSL上では,文字コードをUTF-8に統一できるため,Windows環境で発生しがちな文字コードの不一致問題を解消できる.また,Linux環境のツールを直接利用可能であり,{\TeX} Liveのパッケージ管理もスムーズに行える.ただし,WSLはコマンドラインベースの操作が中心であり,VSCodeと組み合わせることでエディタ機能の拡張が必要である.

\subsection{VSCode DevContainerを用いた環境構築の自動化}
Dockerは,軽量なコンテナ仮想化技術を提供し,ソフトウェアや環境の依存関係をコンテナとしてパッケージ化することができる.Dockerコンテナ上で{\TeX} Liveを構築することで,任意の環境で{\TeX}を実行可能にし,環境設定の手間を大幅に削減することができる.また,複数のシステム間でコンテナを利用することで,再現性の高い{\TeX}環境が確保される.ただし,コンテナ上での編集作業は主にコマンドライン操作となるため,エディタとの連携やGUIを用いた操作性の向上には工夫が必要である.

\subsection{DevContainerを用いた仮想環境の自動構築}
Visual Studio Code(VSCode)のDevContainer機能は,Dockerコンテナを用いて開発環境を自動化するための設定を提供する.DevContainerを利用することで,VSCode上で{\TeX}環境を統合的に利用できるようになり,GUI操作によるファイル編集やビルドが簡単に行える.また,DevContainerの設定ファイルに必要な依存関係やツールを記述することで,特定の環境に依存せずに容易に{\TeX}環境を再現することができる.従来のWSLやDockerと比較して,DevContainerは環境構築の手順が簡便で,ユーザビリティも高い.

\section{まとめ}
これらの関連技術を比較すると,WSL,Docker,DevContainerの順に,仮想化の機能が充実し,ユーザビリティも向上している.WSLは軽量であり,Linux環境を手軽に利用できる利点があるが,操作はコマンドラインに限られる.Dockerは環境の再現性と移植性に優れており,様々なホスト環境で一貫した{\TeX}環境を提供できる.DevContainerは,VSCodeとの統合により直感的なGUI操作が可能となり,初心者でも手軽に{\TeX}環境を構築できる.DevContainerは現時点で最もユーザーフレンドリーで高機能な{\TeX}執筆環境を提供している.
\end{document}